\documentclass{article}

\usepackage{fontspec}
\usepackage{hyperref}

\title{Software Development Plan}
\author{
       Maxime Arthaud
  \and Korantin Auguste
  \and Martin Carton
  \and Etienne Lebrun
  \and Pierre-Louis Michel
}

\begin{document}
  \maketitle
  \tableofcontents
  \newpage

  \section{Introduction}
    This project, suggested by Daniel Hagimont, is based on the CRAPS processor
    developed by Jean-Christophe Buisson and used in the first-year CPU
    architecture courses at ENSEEIHT. The goal is to develop an operating
    system that would run on top of that processor. 

    The reasons for that project are that before, it was only possible to do a
    little of assembly directly on the processor to see it work, but nothing
    more.  After our project, it should be possible for students to really see
    the layer that goes on top of the CPU in modern computers: the operating
    system. So that the students can really make the link between the processor
    they just built and the computer and underlying operating systems they use
    everyday.


  \section{Project Overview}
    The objective of the project is to create an operating system, with a
    scheduler running a few tasks. It will also provide functions to display
    text to the user, do input/output to a permanent storage\dots

    For now, having the OS loading programs dynamically is out-of-scope : the
    goal is to have a very simple functionnal os.  We will also have to improve
    the CPU to make it support our OS. Specifically we know we will have to
    support the RAM chips that are on the FPGA : currently, we have 2 Ko of RAM
    that are built in, and that's clearly not enough.  Another important and
    time-consuming element will be to reuse the compiler we made last year
    during a project, and adapt it to generate the CRAPS assembly. We will
    surely need to make other modifications to the compiler.

  \section{Method, Tools and Test Means}
    \paragraph{Tests} are quite a though point, as our only mean of real test is
    to put our code on the board and try it. That's why we want to proceed
    iteratively and start with very basic things.

    At the end, we want to be able to run a few processes that will communicate
    to the user, and store data in permanent memory.

  \section{Software Team Organisation and Responsibilities}
    \begin{itemize}
      \item Martin Carton as \textit{project leader}
      \item Korantin Auguste
      \item Maxime Arthaud
      \item Etienne Lebrun
      \item Pierre-Louis Michel
    \end{itemize}

  \section{Project Monitoring and Controls}
    Gantt diagram, "hits" that we need to achieve.

    Controls: meetings?

  \section{Meetings and reporting}
    See the "Meetings" folder.

  \section{Management of actions}

  \section{Management of risks}
    \begin{itemize}
      \item A FPGA can be damaged by us, making any test impossible. | 0.5
        (likely) * 0.75 (really damageable) | 0.375 | Have more than one FPGA,
        and have the possibility to have more.
      \item We may not be able to integrate the RAM, leading to huge memory
        limitations that may make the project impossible | 0.2 * 0.9 | 0.18 |
        Put as much ram as we can in the FPGA, in VHDL (but we will have a lot
        less anyway)
    \end{itemize}

  \section{Management of change requests}

  \section{Quality and Configuration management}
    Use of a version control system: a Git repository on Github.

  \section{Documentation}
    \begin{itemize}
      \item User Manual
      \item modified CRAPS
    \end{itemize}
\end{document}
